\rowcolors{2}{white}{gray!25}
\begin{tabularx}{\textwidth}{@{}
		>{\centering\arraybackslash\hsize=.25\hsize\linewidth=\hsize}X
		X
		>{\hsize=1.5\hsize\linewidth=\hsize}X
		X
		>{\hsize=1.5\hsize\linewidth=\hsize}X
		>{\centering\arraybackslash\hsize=.5\hsize\linewidth=\hsize}X
		@{}
	}
	\caption{Tabella dei risultati della valutazione euristica condotta sul sito del comune di Taranto da Andrea Esposito.}
	\label{tab:val-euristica-AndreaEsposito}\\
	\toprule
	N.ro & Locazione & Problema & Euristica violata & Possibile soluzione & Grado di severità\footnotemark \\* \midrule
	\endfirsthead
	\toprule
\rowcolor{white}
	N.ro & Locazione & Problema & Euristica violata & Possibile soluzione & Grado di severità \\* \midrule
	\endhead
	1 & Home page & I bottoni presenti nella sezione ``Servizi'' non sono realmente bottoni ma link & Visibilità dello stato del sistema & Trasformare i link in bottoni, rendendo cliccabile anche il loro sfondo & 4 \\
	2 & Home page & Nella sezione ``Servizi'', alcuni bottoni sono in realtà dei link doppi, ovvero aventi target differenti in base alla zona cliccata. & Visibilità dello stato del sistema & Suddividere i due link in due bottoni differenti & 5 \\
	3 & Pagina ``Bandi di concorso'' & Il link per accedere ai bandi precedenti al 28/11/2019 fa uscire l'utente dal sito senza alcun avvertimento & Riconoscimento piuttosto di memorizzazione & Variare l'icona del link o introdurre un messaggio pop-up che avvisi l'utente & 3 \\
	4 & Home page & La sezione ``Notizie'' (ritenute fondamentale) non è raggiungibile senza fare scrolling & Visibilità dello stato del sistema & Riduzione della dimensione del banner del sito o spostamento della sezione notizie in una sidebar & 3 \\
	5 & Pagina ``Amministrazione e uffici'' & I bottoni non sono realmente bottoni ma link & Visibilità dello stato del sistema & Trasformare i link in bottoni, rendendo cliccabile anche il loro sfondo & 4 \\
	6 & Pagina ``Contatti'' & I bottoni non sono realmente bottoni ma link & Visibilità dello stato del sistema & Trasformare i link in bottoni, rendendo cliccabile anche il loro sfondo & 4 \\
	7 & Pagina ``Contatti'' & Lo stile dei ``bottoni'' non è coerente con quello delle altre pagine & Coerenza e standard & Variare lo stile dei ``bottoni'' in ``Contatti'' per uniformarlo a quello delle altre pagine & 3 \\
	8 & Pagina ``News'' & Il bottone ``Visualizza altri articoli'' suggerisce la pressione del tasto ``shift'' per visualizzare tutti gli articoli, ma è comunque richiesto un click con il mouseda parte dell'utente & Allineamento tra il mondo del sistema e quello reale & Variare la label del bottone affinché suggerisca anche il click & 2 \\
	9 & \textit{Nessuna locazione} & La sezione ``Tasse e tributi locali'' non è raggiungibile senza utilizzare la funzione di ricerca & - & Riprogettare la pagina ``Aree tematiche'' per includere i link mancanti & 5 \\
	10 & Pagina ``Vivere Taranto'' & I bottoni non sono realmente bottoni ma link & Visibilità dello stato del sistema & Trasformare i link in bottoni, rendendo cliccabile anche il loro sfondo & 4 \\
	11 & Pagina ``Teatro, cinema e sale congressi'' & I bottoni non sono realmente bottoni ma link & Visibilità dello stato del sistema & Trasformare i link in bottoni, rendendo cliccabile anche il loro sfondo & 4 \\
	12 & Home page & Le icone legate ad alcuni bottoni sono poco chiare (per esempio, l'icona del bottone ``Statuto e Regolamenti dell'Ente'' & Allineamento tra il mondo del sistema e quello reale & Variare le icone poco chiare & 2 \\
	13 & Home page & L'icona del bottone ``Amministrazione Trasparente'' è una lente d'ingrandimento, associata alla funzione di ricerca) & Coerenza e standard  & Variare l'icona del bottone ``Amministrazione trasparente'' & 3 \\
	14 & Home page & Nella sezione ``Aree tematiche'' vi è un uso di diversi colori che non aggiungono alcuna informazione aggiuntiva & Design estetico e minimalista & Ridurre l'uso inutile di colori aggiuntivi & 1 \\*
	\bottomrule
\end{tabularx}
\footnotetext{Scala $\left [1, 5 \right ]$, dove $1$ indica un problema lieve e $5$ un problema grave}